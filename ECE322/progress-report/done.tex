\subsection{Unit Tests}
Some unit tests were created in a white box testing manner.
These tests were written using JUnit.
\url{https://developer.android.com/training/testing/unit-testing/local-unit-tests}

Currently we have unit tests testing the constructor of MoodEvents, as well as several logical methods from converting database strings to moods.



\subsection{Integration Tests}

Most of the functionality of our app is displaying data from a database,
testing how things are displayed is hard to do in unit tests. So, for this we
used integration tests. For the most part, a automated user interface testing
tool called Espresso was used.
\url{https://developer.android.com/training/testing/espresso}

Currently, we have tests for the app's functionality to add a mood event.
We check each case of adding a mood event, making sure that an error
dialog is shown if not enough information is shown, and testing the
different optional inputs. For example, we have a test for just selecting
the mood, another which does that and adds a comment. There are also tests
for adding moods where the user's location is recorded, and where the test
selects the number of people the user was hanging out with.
These tests assert that the right data is shown on the main screen
after entering the moods. For example, if the user entered `HAPPY', that
must be shown. Additionally, if the user selected `ANGRY', and entered
a comment, that comment should also be shown. If the location was entered,
then NaN should not be shown as a coordinate.

At the end of each test, the mood is deleted, so the delete functionality
is also covered by these tests.

Another functionality our app has is filtering the UI to show only moods of a
specific type. To test this, we have an integration test that does this. This
test will create a 21 moods of various types then will return to the mood
viewing screen. At this point it opens up the filter UI, and selects a filter
state. It then compares the actual moods displayed to the number of moods it
expected to be displayed. It will then repeat this 7 times will different
filter states. It does make sure every single mood type is visible at least
once.


\subsection{User Acceptance Testing}

The user acceptance testing took the form of a live demo.
We have done a bunch of user acceptance testing. A large part of our manual
testing has been in this form. The functionality for adding moods, deleting
moods, editing moods, viewing moods, and filtering moods have all passed user
acceptance testing.



